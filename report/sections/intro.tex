

% ==============================================================================
% START: Introduction
% ==============================================================================
\section{Introduction}

The application is the last requirement to graduate from Udacity's Data Engineering Nano degree. The following is
the overview of the project as stated on Udacity page course. \\

\emph{The purpose of the data engineering capstone project is to give you a chance to 
combine what you've learned throughout the program. This project will be an important part of 
your portfolio that will help you achieve your data engineering-related career goals.}


\subsection{My vision}
The project has two big parts. The backend was designed to run on the cloud. From storage on S3, both the raw files and the parquet model
to EMR processing and Relational database on Redshift. This would allow various applications to be served by this back-end. The other part would be
the front end, on local/client side, this means that local user is expected to have the web interface on flask. This would allow him/her to query
the Redshift database plus the added capability of asking a question on papers of interest. I image a local user, setting up the flask app, and then exploring the database with it, then later, indexing on to his local elastic-search server, the abstract and title of paper of interests, this could mean
filtering by a particular topic, year, or author. After the indexing, the user can ask questions of the type \emph{what do we know about the uncertainty principle?}. The answer will come on the form of a bootstrap card, with scores, possible answers, full paper URL pdf access to the document. The question and answering system is possible with \href{https://github.com/deepset-ai/haystack}{haystack}. The following are the core features of haystack on their github.




\begin{itemize}
\item Powerful ML models: Utilize all latest transformer based models (BERT, ALBERT, RoBERTa ...)
\item Modular \& future-proof: Easily switch to newer models once they get published.
\item Developer friendly: Easy to debug, extend and modify.
\item Scalable: Production-ready deployments via Elasticsearch backend \& REST API
\item Customizable: Fine-tune models to your own domain \& improve them continuously via user feedback
\end{itemize}


    
    
    
    


% ==============================================================================
% START: Introduction
% ==============================================================================